\documentclass{article}

\title{Effect Network \\ \vspace{16pt} \large \textbf{DRAFT}}
\date{\today}
\author{Jesse Eisses, Laurens Verspeek \\
  \small \texttt{\{jeisses, lverspeek\}@itsavirus.com}}

\begin{document}

\maketitle

\begin{abstract}

\end{abstract}


\section{Introduction}

In the past half-decade there has been a rapid growth in the number of
practical Artificial Intelligence applications around us. Smart
devices, like self driving cars, face and voice recognition in mobile
phones and image translation are getting a central place in everyday
life. This rise can be explained by the recent advances in machine
learning research in combination with the large adoption from the
industry. While the academic achievements are available to the public,
most applied AI algorithms are developed and run at large corporations
behind closed doors. Three major reasons that make the development
inaccessible for individuals is given below:

\begin{description}
\item[Data processing] A common property of intelligent applications
  is that they perform tasks that traditionally require human
  feedback. Such tasks involve processing unstructured data and
  finding a pattern that can be used to provide useful output. These
  application are trained on large datasets with annotations. Obtaining
  an annotated dataset is non-trivial and requires a lot of time and
  money.
\item[Diverging tasks] An obstacle when developing a complex
  algorithm is the need to interact with parts of the world outside
  the algorithms domain. For example, a self driving car will need to
  identify road signs around the world. Practice teaches us that these
  situations can best be treated as a knowledge system where the
  classification of the sign is done by an external application. This
  is a large burden for small
\item[Computational cost] Developing and training a large AI is a
  computational intensive task, and requires a technical
  infrastructure capable of processing terabytes of data, doing
  batched processing on multiple GPUs and coordinating the results.
\end{description}

To make the development of AI applications more accessible, and to
stimulate their growth and freedom, we propose a private,
decentralized ecosystem called the \emph{Effect Network}. The network
is designed to provide a fully decentralized alternative to the
markets shown in table~\ref{tab:service_compare}. The decentralization
is achieved by running on a blockchain that supports Turning-complete
smart contracts.

\begin{table}
  \centering
  \begin{tabular}[h]{l|l|l}
    \textbf{Market} & \textbf{Suppliers} & \textbf{Market Cap.} \\ \hline
    Micro tasking & Amazone Mechenical Turk, Fiverr, \dots & \dots \\ 
    AI as a service & Watson, Amazon Rekognition, \dots & \dots \\
    Computational platform & Google Cloud ML, Amazone AI, \dots & \dots 
  \end{tabular}
  \caption{Overview of markets (WIP)}\label{tab:service_compare}
\end{table}

The \emph{Effect Network}, like other decentralized applications,
directly connects supply and demand without the need for an
intermediary party. Competitive prices \dots

\section{Products}

The \emph{Effect Network} consists of 3 main products/platforms?

\subsection{Human Intelligence Tasks}

\subsection{Effect Network?}

\subsection{Computational Platform}

\section{Community}

\subsection{AIX token}

\subsection{Quality and Fraud}

\subsection{Governance}

\section{Examples}

\section{Note on Ethics}

\end{document}
